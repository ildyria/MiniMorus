%!TEX root = morusAC.tex

\section{Preliminaries}
\label{sec/Preliminaries}

\MORUS is a family of authenticated ciphers designed by Wu and Huang \cite{MORUS}.
An instance of \MORUS is parametrized by a secret key $K$. During encryption, it takes as input a plaintext message $M$, a nonce $N$, and possibly some associated data $A$, and outputs a ciphertext $C$ together with an authentication tag $T$.
In this section, we provide a brief description of \MORUS and introduce the
notation for linear approximations.

\subsection{Specification of \texorpdfstring{\MORUS}{MORUS}}
\label{subsec/Spec}

The \MORUS family supports two internal state sizes: 640 and 1280 bits,
referred to as \MORUS[640] and \MORUS[1280], respectively.
Three parameter sets are recommended: \MORUS[640] supports 128-bit keys and \MORUS[1280] supports either 128-bit or 256-bit keys. The tag size is 128 bits or shorter. The designers strongly recommend using a 128-bit tag.
%
%\subsubsection{Security.}
With a 128-bit tag, integrity is claimed up to 128 bits and confidentiality is claimed up to the number of key bits (\Cref{Tbl/security}).

\begin{table}[b]
\caption{Security goals of \MORUS.} \label{Tbl/security}
\centering
\begin{tabular}{@{}l@{\quad}c@{\qquad}c@{}}\toprule
                        & Confidentiality (bits) & Integrity (bits) \\ \midrule
\MORUS[640-128]   & 128                    & 128              \\
\MORUS[1280-128]  & 128                    & 128              \\
\MORUS[1280-256]  & 256                    & 128              \\ \bottomrule
\end{tabular}
\end{table}

\subsubsection{State.}
The internal state of \MORUS is composed of five $q$-bit \emph{registers} $S_i$, $i \in\{0,1,2,3,4\}$, where $q = 128$ for \MORUS[640] and $q=256$ for \MORUS[1280]. The internal state of \MORUS may be represented as $S_0\|S_1\|S_2\|S_3\|S_4$.
Registers are themselves divided into four $q/4$-bit \emph{words}.
Throughout the paper, we denote the word size by $w = q/4$, i.e., $w=32$ for \MORUS[640] and $w=64$ for \MORUS[1280].

The encryption process of \MORUS consists of four parts: initialization, associated data processing, encryption, and finalization. 
During the initialization phase, the value of the state is initialized using a key and nonce.
The associated data and the plaintext are then processed block by block.
Then the internal state undergoes the finalization phase, which outputs the authentication tag.

Every part of this process relies on iterating the \StateUpdate{} function at the core of \MORUS. Each call to the \StateUpdate{} function is called a step.
%In order to denote the internal state of the cipher at the $t$-th step, we use a superscript to denote the step number:
The internal state at step $t$ is denoted by $S^t_0\|S^t_1\|S^t_2\|S^t_3\|S^t_4$, where $t = -16$ before the initialization and $t=0$ after the initialization.

%Hereafter we explain \StateUpdate{}.

\subsubsection{The \StateUpdate{} Function.}
\StateUpdate{} takes as input the internal state $S^t = S^t_0\|S^t_1\|S^t_2\|S^t_3\|S^t_4$ and an additional $q$-bit value $m^t$ (recall that $q$ is the size of a register), and outputs an updated internal state.
%During the associated data processing phase and encryption phase, $m^t$ is is an associated data block or message block. Otherwise $m^t$ can be 0 or constant when the additional input is unnecessary.

\StateUpdate{} is composed of 5 rounds with similar operations.
The additional input $m^t$ is used in rounds 2 to 5, but not in round 1.
Each round uses the bit-wise rotation (left circular shift) operation inside word,
denoted $\lll_w$ in the following and \texttt{Rotl\_xxx\_yy} in the design document.
It divides a $q$-bit register value into 4 words of $w = q/4$ bits, and performs a rotation on each $w$-bit word.
The bit-wise rotation constants $b_i$ for round $i$ are defined in \Cref{Tbl:rcon}.
Additionally, each round uses rotations on a whole $q$-bit register by a multiple of the word size,
denoted $\lll$ in the following and \texttt{<<<} in the design document.
The word-wise rotation constants $b_i'$ are also listed in \Cref{Tbl:rcon}.
%\todo{rounds $0\dots4$ or $1\dots5$?}

$S^{t+1} \leftarrow \StateUpdate{}(S^t,m^t)$ is defined as follows, where $\cdot$ denotes bit-wise AND, $\oplus$ is bit-wise XOR, and $m_i$ is defined depending on the context:
\begin{align*}
\textrm{Round 1:} &&
S^{t+1}_0 &\leftarrow ( S^t_0 \oplus (S^t_1 \cdot S^t_2) \oplus S^t_3 )            \lll_w b_0, &
S^t_3 \leftarrow S^t_3 \lll b'_0.\\
\textrm{Round 2:} &&
S^{t+1}_1 &\leftarrow ( S^t_1 \oplus (S^t_2 \cdot S^t_3) \oplus S^t_4 \oplus m_i ) \lll_w b_1, &
S^t_4 \leftarrow S^t_4 \lll b'_1.\\
\textrm{Round 3:} &&
S^{t+1}_2 &\leftarrow ( S^t_2 \oplus (S^t_3 \cdot S^t_4) \oplus S^t_0 \oplus m_i ) \lll_w b_2, &
S^t_0 \leftarrow S^t_0 \lll b'_2.\\
\textrm{Round 4:} &&
S^{t+1}_3 &\leftarrow ( S^t_3 \oplus (S^t_4 \cdot S^t_0) \oplus S^t_1 \oplus m_i ) \lll_w b_3, &
S^t_1 \leftarrow S^t_1 \lll b'_3.\\
\textrm{Round 5:} &&
S^{t+1}_4 &\leftarrow ( S^t_4 \oplus (S^t_0 \cdot S^t_1) \oplus S^t_2 \oplus m_i ) \lll_w b_4, &
S^t_2 \leftarrow S^t_2 \lll b'_4.
\end{align*}

\begin{table}[t]%[!bht]
  \centering
  \caption{Rotation constants $b_i$ for $\lll_w$ and $b_i'$ for $\lll$ in round $i$ of \texorpdfstring{\MORUS}{MORUS}.} \label{Tbl:rcon}
  \setlength{\tabcolsep}{6pt}
  \begin{tabular}{@{}l@{\qquad}rrrrr rrrrr@{}}
    \toprule
    & \multicolumn{5}{@{}c@{~}}{Bit-wise rotation $\lll_w$} & \multicolumn{5}{@{}c@{}}{Word-wise rotation $\lll$} \\
    \cmidrule(r){2-6}
    \cmidrule(l){7-11}
    & $b_0$  & $b_1$  & $b_2$  & $b_3$  & $b_4$  
    & $b_0'$ & $b_1'$ & $b_2'$ & $b_3'$ & $b_4'$  \\
    %& \makebox[2em]{$b_0$} & \makebox[2em]{$b_1$} & \makebox[2em]{$b_2$} & \makebox[2em]{$b_3$} & \makebox[2em]{$b_4$} & \makebox[2em]{$b'_0$} & \makebox[2em]{$b'_1$} & \makebox[2em]{$b'_2$} & \makebox[2em]{$b'_3$} & \makebox[2em]{$b'_4$} \\ \hline
    \midrule
    \MORUS[640]  &  5 & 31 &  7 & 22 & 13 & 32 &  64 &  96 &  64 & 32 \\
    \MORUS[1280] & 13 & 46 & 38 &  7 &  4 & 64 & 128 & 192 & 128 & 64 \\
    \bottomrule
  \end{tabular}
\end{table}

\subsubsection{Initialization.}
The initialization of \MORUS[640] starts by loading the 128-bit key $K_{128}$ and the 128-bit nonce $N_{128}$ into the state together with constants $c_0, c_1$:
%and running \StateUpdate{} for 16 steps. The key and nonce are loaded into the state as follows:
\begin{align*}
S^{-16}_0 &= N_{128}, &
S^{-16}_1 &= K_{128}, &
S^{-16}_2 &= 1^{128}, &
S^{-16}_3 &= c_0, &
S^{-16}_4 &= c_1.
\end{align*}
Then, $\StateUpdate{}(S^t,0)$ is iterated 16 times for $t=-16,-15,\ldots,-1$.
Finally, the key is XORed into the state again with $S^0_1 \leftarrow S^0_1 \oplus K_{128}$.

The initialization of \MORUS[1280] differs slightly due to the difference in register size and the two possible key sizes,
and uses either
$K = K_{128} \|K_{128}$ (for \MORUS[1280-128]) or
$K = K_{256}$ (for \MORUS[1280-256])
to initialize the state:
\begin{align*}
S^{-16}_0 &= N_{128} \mathrel\| 0^{128}, &
S^{-16}_1 &= K, &
S^{-16}_2 &= 1^{256}, &
S^{-16}_3 &= 0^{256}, &
S^{-16}_4 &= c_0\mathrel\|c_1.
\end{align*}
After iterating \StateUpdate{} 16 times, the state is updated with $S^0_1 \leftarrow S^0_1 \oplus K$.


\subsubsection{Associated Data Processing.}
After initialization, the associated data $A$ is processed in blocks of $q \in \{128, 256\}$ bits.
For the padding, if the last associated data block is not a full block, it is padded to $q$ bits with zeroes.
If the length of $A$, denoted by $|A|$, is 0, then the associated data processing phase is skipped;
else, the state is updated as
\[S^{t+1} \leftarrow \StateUpdate{}(S^t, A^t) \qquad \text{for } t=0,1,\ldots,\lceil |A|/q\rceil-1.\]

\subsubsection{Encryption.}
Next, the message is processed in blocks $M_t$ of  $q \in \{128, 256\}$ bits to update the state and produce the ciphertext blocks $C_t$.
If the last message block is not a full block, a string of 0's is used to pad it to 128 or 256 bits for \MORUS[640] and \MORUS[1280], respectively, and the padded full block is used to update the state. However, only the partial block is encrypted. Note that if the message length denoted by $|M|$ is 0, encryption is skipped.
Let $u = \lceil |A|/q \rceil$ and $v = \lceil |M|/q \rceil$. The following is performed for $t=0, 1, \ldots, v-1$:
\begin{align*}
C^t &\leftarrow M^t \oplus S^{u+t}_0 \oplus (S^{u+t}_1 \lll b_2') \oplus (S^{u+t}_2 \cdot S^{u+t}_3),\\
S^{u+t+1} &\leftarrow \StateUpdate{}(S^{u+t},M^t).
\end{align*}

\subsubsection{Finalization.}
The finalization phase generates the authentication tag $T$ using 10 more \StateUpdate{} steps. We only discuss the case where $T$ is not truncated. The associated data length and the message length are used to update the state:
\begin{enumerate}
\item $L \leftarrow |A| \mathrel\| |M|$ for \MORUS[640] or $L \leftarrow |A| \mathrel\| |M| \mathrel\|0^{128}$ for \MORUS[1280], where $|A|, |M|$ are represented as 64-bit integers.
\item $S^{u+v}_4 \leftarrow S^{u+v}_4 \oplus S^{u+v}_0.$
\item For $t = u+v, u+v+1, \ldots, u+v+9,$ compute
%\begin{equation*}
$S^{t+1} \leftarrow \StateUpdate{} (S^t, L)$.
%\end{equation*}
%\item For \MORUS[640], $T = S^{u+v+10}_0 \oplus (S^{u+v+10}_1 \lll 96) \oplus ( S^{u+v+10}_2 \cdot S^{u+v+10}_3)$. For \MORUS[1280], $S^{u+v+10}_0 \oplus (S^{u+v+10}_1 \lll 192) \oplus ( S^{u+v+10}_2 \cdot S^{u+v+10}_3)$ is computed and the least significant 128 bits are produced as a tag $T$.
\item $T = S^{u+v+10}_0 \oplus (S^{u+v+10}_1 \lll b_2') \oplus ( S^{u+v+10}_2 \cdot S^{u+v+10}_3)$, or the least significant 128 bits of this value in case of \MORUS[1280].
\end{enumerate}

\subsection{Notation}
\label{subsec/Notation}

In the following, we use linear approximations \cite{eurocryptMatsui93}
that hold with probability $\Pr(E) = \frac{1}{2} + \varepsilon$, i.e., they are biased with bias $\varepsilon$.
The \emph{correlation} $\corr(E)$ of the approximation and its \emph{weight} $\weight(E)$ are defined as
\begin{align*}
\corr(E) &\eqdef 2\Pr(E)-1 = 2 \varepsilon\,, \\
%We also use the \emph{weight} of the approximation, defined as
\weight(E) &\eqdef -\log_2|\corr(E)|\,,
\end{align*}
where $\log_2()$ denotes logarithm in base 2. By the Piling-Up Lemma, the correlation (resp. weight) of an XOR of independent variables is equal to the product (resp. sum) of their individual correlations (resp. weights) \cite{eurocryptMatsui93}.

We also recall the following notation from the previous section, where an \emph{encryption step} refers to one call to the \StateUpdate{} function:\\
\centerline{
\def\arraystretch{1.25}
\begin{tabular}{lcl}
$C^t$ &:& the ciphertext block output during the $t$-th encryption step.\\
$C^t_j$ &:& the $j$-th bit of $C^t$, with $C^t_0$ being the rightmost bit.\\
$S^t_i$ &:& the $i$-th register at the beginning of $t$-th encryption step.\\
$S^t_{i,j}$ &:& the $j$-th bit of $S^t_i$, with $S^t_{i,0}$ being the rightmost bit.
\end{tabular}
}
In the above notation, bit positions are always taken modulo the register size $q$, i.e., $q=128$ for \MORUS[640] and $q=256$ for \MORUS[1280].

For simplicity, in the remainder, the $0$-th encryption step will often denote the encryption step where our linear trail starts. Any encryption step could be chosen for that purpose, as long as at least four more encryption steps follow. In particular the $0$-th encryption step from the perspective of the trail does not have to be the first encryption step after initialization.

%Finally, let $x$ denote a register. Recall that registers of \MORUS are composed of four words. We now define word-wise and bit-wise rotations (denoted by \texttt{<<<} and \texttt{Rotl\_xxx\_yy} respectively in the \MORUS design document):\\
%\centerline{
%\def\arraystretch{1.25}
%\begin{tabular}{lcl}
%$x \rotl y$ &:& circular shift of $x$ by $y$ bits to the left.\\
%$x \rotlxy y$ &:& circular shift of each word within $x$ by $y$ bits to the left.\\
%\end{tabular}
%}
%
%The $b_i$'s denote rotation constants, whose value depends on which version of \MORUS is considered. Throughout, $w$ denotes the  word size, i.e. $w=32$ for \MORUS[640] and $w=64$ for \MORUS[1280].

%%% Local Variables:
%%% TeX-master: "morusAC"
%%% End:
