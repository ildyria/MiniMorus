%!TEX root = morusAC.tex

\begin{abstract}
  \MORUS is a high-performance authenticated encryption algorithm
  submitted to the CAESAR competition, and recently selected as a
  finalist.  There are three versions of \MORUS: \MORUS[640] with
  %a 640-bit state and
  a 128-bit key, and \MORUS[1280] with
  % a 1280-bit state and
  128-bit or 256-bit keys. For all versions the security
  claim for confidentiality matches the key size.  In this paper, we
  analyze the components of this algorithm (initialization, state
  update and tag generation), and report several results.

  As our main result, we present a
  linear correlation in the keystream of full \MORUS, which can be used to distinguish its output
  from random and to recover some plaintext bits in the
  broadcast setting.  For \MORUS[1280], the correlation is $2^{-76}$,
  which can be exploited after around $2^{152}$ encryptions, less than
  what would be expected for a 256-bit secure cipher.  For
  \MORUS[640], the same attack results in a correlation of $2^{-73}$, which does not
  violate the security claims of the cipher.

% More precisely, we show that a linear correlation exists between some bits of the plaintext and ciphertext. Moreover the bias depends purely on the plaintext, and not on the secret key nor the nonce, hence restricting the single use of nonce or even rekeying does not prevent the attacks.
%
% From the nature of the linear cryptanalysis, the attack works in the known plaintext attacks. Moreover, if a fixed plaintext is encrypted multiple times, the correlation can be used to recover a 1-bit information of the plaintext. The attack principally has larger impact because it can be used to recover more unknown bits of the plaintext, provided an initial segment of the plaintext is known. 
%
  To identify this correlation, we make use of rotational invariants in \MORUS
  using linear masks that are invariant by word-rotations of the
  state.  This motivates us to introduce single-word versions of
  \MORUS called \MiniMORUS, which simplifies the
  analysis. The attack has been implemented and verified on
  \MiniMORUS, where it yields a correlation of $2^{-16}$.

%The bias for \cipher{MORUS1280} is slightly worse than the 4 copies of \cipher{MiniMORUS}, and the mechanism behind is explained.
%
%Towards the better understanding of \cipher{MORUS}, several other properties are investigated though these do not break any security claim by the designers. Those include the bias for \cipher{MORUS640} requiring $2^{146}$ encryption against 128-bit security, and analyses on initialization and finalization of the step-reduced version of \cipher{MORUS1280}.


%In this note, we show the existence of linear biases in the output of
%the authenticated encryption scheme MORUS. More precisely, we show
%that when encrypting a fixed plaintext multiple times, a linear
%correlation exists between some bits at the output of the
%cipher. Moreover the bias depends purely on the plaintext, and not on
%the secret key of the cipher. In principle, this property could be
%used to recover unknown bits of a plaintext encrypted a large number
%of times, provided an initial segment of the plaintext is
%known. However, the biases are quite small: for the smaller MORUS640
%variant, claiming a security level of 128 bits, the bias is
%$2^{-73}$, and would thus require around $2^{146}$ encryptions to be
%exploited directly; for MORUS1280, claiming a security level of 256
%bits, the bias is $2^{-76}$, corresponding to a data requirement of
%$2^{152}$ encryptions. We note that rekeying does not prevent the
%attack: the biases are independent of the secret encryption key, and
%can be exploited as long as a given plaintext is encrypted enough
%times, regardless of whether each encryption uses a different key.

  We also study reduced versions of the initialization and
  finalization of \MORUS, aiming to evaluate the security margin of these
  components.  We show a forgery attack when finalization is
  reduced from 10 steps to 3, and a key-recovery attack in the
  nonce-misuse setting when initialization is reduced from 16 steps to 10.  These additional results do not threaten the full \MORUS, but
  studying all aspects of the design is useful to understand its
  strengths and weaknesses.

\keywords{MORUS, CAESAR, Authenticated Encryption, Nonce Respecting, Linear Cryptanalysis, Confidentiality.}
\end{abstract}

%%% Local Variables:
%%% TeX-master: "morusAC"
%%% End:
