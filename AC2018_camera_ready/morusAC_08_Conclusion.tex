%!TEX root = morusAC.tex

\section{Conclusion}
\label{sec/Conclusion}

This work provides a comprehensive analysis of the components of
\MORUS.  In particular, we show that \MORUS[1280]'s keystream exhibits a correlation of $2^{-76}$ between certain ciphertext bits. This enables a plaintext recovery
attack in the broadcast setting, using about $2^{152}$ blocks of data.
While the amount of data required is impractical, this seems to
violate the security claims of \MORUS[1280] because the attack works
even if the key is refreshed regularly.  Moreover, the broadcast
setting is practically relevant, as was shown with attacks against RC4
as used in TLS~\cite{DBLP:conf/uss/AlFardanBPPS13}.

We have shared an earlier version of this paper with the authors of
\MORUS and they agree with the technical details of the keystream
bias.  However they consider that it is not a significant weakness in
practice because it requires more than $2^{64}$ ciphertexts bits.  In
the context of the CAESAR competition, we believe that certificational
attacks such as this one should be taken into account, in order to
select a portfolio of candidates that reflects the state of the art in
terms of cryptographic design.

\ifanonymous
\else
\subsubsection{Acknowledgments.}

The results presented here were originally found during the Flexible
Symmetric Cryptography workshop held at the Lorentz Center in Leiden,
Netherlands.  The authors would like to thank Meltem Sonmez Turan, who
participated in the initial discussion.
The second author was supported by the European Union's H2020 grant 644052 (HECTOR).
The fourth and sixth authors
are partially supported by the French Agence Nationale de la Recherche
through the BRUTUS project under Contract ANR-14-CE28-0015.
The fifth author was supported by EPSRC Grant EP/M013472/1.

\fi



%%% Local Variables:
%%% TeX-master: "morusAC"
%%% End:
