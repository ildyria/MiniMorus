\section{Introduction}
\label{sec/Introduction}

Block ciphers provide confidentiality through encryption. Hash functions are the building blocks of integrity and authenticity. These two operations combine in a single scheme class named Authenticated Encryption. From a message; IV and key, it produces a ciphertext and a final authentication tag.

\noindent
The CAESAR\cite{CAESAR} Competition aims to provide with a portfolio of authenticated encryption schemes. The selection is made by comparing the resistances to a wide range of attacks. Morus is one of the last 3 finalist.

\subsubsection*{Our Contributions.}
Our main result is the existence of a linear trail linking ciphertext bits spanning five consecutive encryption steps. As a consequence of this trail, we exhibit a biased linear combination of ciphertext bits, which holds across any given sequence of five consecutive encryption steps. Technically, this assumes that the inner state of the cipher at the start of the trail is uniformly random, but this is a fair assumption since the inner state should be indistinguishable from uniform randomness by design. Moreover, that is the only assumption required; in particular, the secret key or the nonce used to initialize the inner state of MORUS are irrelevant: the bias is independent of their value.

\subsubsection*{Paper Outline.}
This paper is organized as follows, we first provide a brief introduction of how Morus operates\ref{sec/Preliminaries}. By showing existance of a certain class of rotational symmetries, we provide with a reduction of Morus namely MiniMorus\ref{sec/introminimorus}. We analysed this simplified scheme in \ref{sec/minitrails}, building a ciphertext-only trail with a weight of 16.
We then extend our result to the full scheme, showing a bias in the keystream over 5 rounds. In \ref{sec/IniFin} we provide new attacks on reduced version of the initialization and the finalization.
