\begin{abstract}
This paper presents linear biases in the output of an authenticated encryption scheme \cipher{MORUS} that is one of the seven finalists in CAESAR. The bias allows an attacker to distinguish it from an ideal cipher, thus breaks the designer's confidentiality claim.
%
More precisely, we show that a linear correlation exists between some bits of the plaintext and ciphertext. Moreover the bias depends purely on the plaintext, and not on the secret key nor the nonce, hence restricting the single use of nonce or even rekeying does not prevent the attacks.
%
From the nature of the linear cryptanalysis, the attack works in the known plaintext attacks. Moreover, if a fixed plaintext is encrypted multiple times, the correlation can be used to recover a 1-bit information of the plaintext. The attack principally has larger impact because it can be used to recover more unknown bits of the plaintext, provided an initial segment of the plaintext is known. 
%
The bias for \cipher{MORUS1280}, claiming a security level of 256 bits, is $2^{-76}$, would thus be identified by around $2^{152}$ encryptions.
%
To identify the bias, the self-similarity of the \cipher{MORUS1280} operation is exploited. More precisely, \cipher{MORUS1280} operates on 256-bit registers that are further divided into four 64-bit words. Many operations of \cipher{MORUS1280} work in a symmetric-way. This motivates us to introduce word-oriented version of \cipher{MORUS} called \cipher{MiniMORUS}, which helps the analysis. The attack is implemented and verified on \cipher{MiniMORUS} having a bias of $2^{-16}$. 

%The bias for \cipher{MORUS1280} is slightly worse than the 4 copies of \cipher{MiniMORUS}, and the mechanism behind is explained.
%
%Towards the better understanding of \cipher{MORUS}, several other properties are investigated though these do not break any security claim by the designers. Those include the bias for \cipher{MORUS640} requiring $2^{146}$ encryption against 128-bit security, and analyses on initialization and finalization of the step-reduced version of \cipher{MORUS1280}.


%In this note, we show the existence of linear biases in the output of the authenticated encryption scheme MORUS. More precisely, we show that when encrypting a fixed plaintext multiple times, a linear correlation exists between some bits at the output of the cipher. Moreover the bias depends purely on the plaintext, and not on the secret key of the cipher. In principle, this property could be used to recover unknown bits of a plaintext encrypted a large number of times, provided an initial segment of the plaintext is known. However, the biases are quite small: for the smaller MORUS640 variant, claiming a security level of 128 bits, the bias is $2^{-73}$, and would thus require around $2^{146}$ encryptions to be exploited directly; for MORUS1280, claiming a security level of 256 bits, the bias is $2^{-76}$, corresponding to a data requirement of $2^{152}$ encryptions. We note that rekeying does not prevent the attack: the biases are independent of the secret encryption key, and can be exploited as long as a given plaintext is encrypted enough times, regardless of whether each encryption uses a different key.
\keywords{MORUS, CAESAR, Authenticated Encryption, Nonce Respect, Linear Cryptanalysis, Confidentiality.}
\end{abstract}