\begin{abstract}
In this note, we show the existence of linear biases in the output of the authenticated encryption scheme MORUS. More precisely, we show that when encrypting a fixed plaintext multiple times, a linear correlation exists between some bits at the output of the cipher. Moreover the bias depends purely on the plaintext, and not on the secret key of the cipher. In principle, this property could be used to recover unknown bits of a plaintext encrypted a large number of times, provided an initial segment of the plaintext is known. However, the biases are quite small: for the smaller MORUS640 variant, claiming a security level of 128 bits, the bias is $2^{-73}$, and would thus require around $2^{146}$ encryptions to be exploited directly; for MORUS1280, claiming a security level of 256 bits, the bias is $2^{-76}$, corresponding to a data requirement of $2^{152}$ encryptions. We note that rekeying does not prevent the attack: the biases are independent of the secret encryption key, and can be exploited as long as a given plaintext is encrypted enough times, regardless of whether each encryption uses a different key.
\keywords{Linear Cryptanalysis, MORUS, CAESAR.}
\end{abstract}