%!TEX root = morusAC.tex

\section{Conclusion}
\label{sec/Conclusion}

This work provides a comprehensive analysis of the components of
\MORUS.  In particular we show that the state update has important
weaknesses with a linear bias of $2^{-76}$ between ciphertext bits and
plaintext bits in \MORUS[1280].  This enables a plaintext recovery
attack in the broadcast setting, using about $2^{152}$ blocs of data.
While the amount of data required is impractical, this seems to
violate the security claims of \MORUS[1280] because the attack works
even if the key is refreshed regularly.  Moreover, the broadcast
setting is practically relevant, as was shown with attacks aginst RC4
as used in TLS~\cite{DBLP:conf/uss/AlFardanBPPS13}.

We have shared an earlier version of this paper with the authors of
\MORUS and they agree with the technical details of the keystream
bias.  However they consider that it is not a significant weakness in
practice because it requires more than $2^{64}$ ciphertexts bits.  In
the context of the CAESAR competition, we believe that certificational
attacks such as this one should be taken into account, in order to
select a portfolio of candidates that reflects the state of the art in
terms of cryptographic design.


%%% Local Variables:
%%% TeX-master: "morusAC"
%%% End:
