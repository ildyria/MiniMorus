\section{Discussion}

%This is just a sketch for now...

We emphasize that the biases we uncover between ciphertext bits are \emph{absolute}, in the sense that they do not depend on the encryption key, or the nonce. As such, they could be leveraged to mount an attack in a known-prefix, unknown-suffix scenario, similar to biases on the RC4 stream cipher. This is especially effective in a setting where the attacker is able to force multiple encryptions of the same plaintext, while controlling the length of the known prefix, such as in the BEAST attack on TLS.

The design document of \cipher{MORUS} imposes a limit of $2^{64}$ encrypted blocks for a given key. However, since our attack is independent of the encryption key, and hence immune to rekeying, this limitation does not apply: all that matters for our attack is that the same plaintext be encrypted enough times.

With the trail presented in this work, the data complexity is still out of reach, since exploiting the bias would require $2^{146}$ encrypted blocks for \cipher{MORUS640}, and $2^{152}$ encrypted blocks for \cipher{MORUS1280}. The data complexity could be slightly lowered by leveraging multilinear cryptanalysis; indeed, the trail holds for any bit shift, and if we assume independence, we could run $w$ copies of the trail in parallel on the same encrypted blocks (recall that $w$ is the word size, and the trail is invariant by rotation by $w$ bits). This would save a factor $2^5$ on the data complexity for \cipher{MORUS640}, and $2^6$ for \cipher{MORUS1280}; but the resulting complexity is still out of reach.

However, the existence of this trail does hint at some weakness in the design of \cipher{MORUS}. Indeed, a notable feature of the trail is that the values of rotation constants are mostly irrelevant: a similar trail would exist for most choices of the constants. That it is possible to build a trail that ignores rotation constants may be surprising. This would have been prevented by adding a word-wise rotation to one of the state registers at the input of the ciphertext equation.
