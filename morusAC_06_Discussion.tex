\section{Discussion}
\label{sec/Discussion}

%This is just a sketch for now...

We now discuss the impact of our attacks for the security of \morus.

\subsubsection{Keystream biases.}

We emphasize that the biases we uncover between ciphertext bits are
\emph{absolute}, in the sense that they do not depend on the encryption
key, or the nonce.  This is the same situation as the keystream biases
in AEGIS~\cite{sacryptMinaud14}.  As such, they can be leveraged to
mount an attack in the broacast setting, where the same message is
encrypted multiple times with different IVs and potentially different
keys~\cite{DBLP:conf/fse/MantinS01}.  In particular, the broadcast setting appears
in practice in man-in-the-browser attacks against HTTPS connections
following the BEAST model~\cite{duong2011here}.  In this scenario, an
attacker uses Javascript code running in the victim's browser (by
tricking the victim to visit a malicious website) to generate a large
number of request to a secure website.  Because of details of the HTTP
protocol, each request includes an authentication token to identify the
user, and the attacker can target this token as a repeated plaintext.
Concretely, biases in the RC4 keystream have been exploited in this
setting, leading to the recovery of authentication cookies in
practice~\cite{DBLP:conf/uss/AlFardanBPPS13}.

\subsubsection{Data complexity.}

The design document of \cipher{MORUS} imposes a limit of $2^{64}$
encrypted blocks for a given key. However, since our attack is
independent of the encryption key, and hence immune to rekeying, this
limitation does not apply: all that matters for our attack is that the
same plaintext be encrypted enough times.

With the trail presented in this work, the data complexity is clearly
out of reach in practice, since exploiting the bias would require
$2^{152}$ encrypted blocks for \cipher{MORUS1280}, and $2^{146}$
encrypted blocks for \cipher{MORUS640}. The data complexity could be
slightly lowered by leveraging multilinear cryptanalysis; indeed, the
trail holds for any bit shift, and if we assume independence, we could
run $w$ copies of the trail in parallel on the same encrypted blocks
(recall that $w$ is the word size, and the trail is invariant by
rotation by $w$ bits). This would save a factor $2^5$ on the data
complexity for \cipher{MORUS640}, and $2^6$ for \cipher{MORUS1280}; but
the resulting complexity is still out of reach.

However, \cipher{MORUS1280} with a 256-bit key claims a security level
of 256 bits for confidentiality, and an attack with complexity $2^{152}$
should violate this claim, even if it is not practical.  The
\cipher{MORUS} specification limits the amount of plaintext encrypted
with a given key to $2^{64}$, but this does not affect our attack
because we can use keystream generated with different keys.

Moreover, the existence of this trail does hint at some weakness in the
design of \cipher{MORUS}. Indeed, a notable feature of the trail is that
the values of rotation constants are mostly irrelevant: a similar trail
would exist for most choices of the constants. That it is possible to
build a trail that ignores rotation constants may be surprising. This
would have been prevented by adding a word-wise rotation to one of the
state registers at the input of the ciphertext equation.

%%% Local Variables:
%%% TeX-master: "morusAC"
%%% End:
