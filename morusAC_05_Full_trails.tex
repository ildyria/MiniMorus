%!TEX root = morusAC.tex

\section{Trail for Full \MORUS}
\label{sec/fulltrails}

In the previous section, we presented a linear trail for the reduced ciphers \MiniMORUS[1280] and \MiniMORUS[640]. We now turn to the full ciphers \MORUS[1280] and \MORUS[640].

\subsection{Symmetrizing the Trail}
In order to build a trail for the full \MORUS, we proceed exactly as we did for \MiniMORUS, following the same path down to step and word rotation values, with one difference: in order to move from the one-word registers of \MiniMORUS to the four-word registers of full \MORUS, we make every term $S^t_{i,j}$ and $C^t_j$ symmetric, in the sense of \Cref{sec/introminimorus}.
That is, for every $S^t_{i,j}$ (resp. $C^t_j$) component in every trail fragment and every equation, we \emph{symmetrize} the term by adding in the terms $S^t_{i,j+w}$, $S^t_{i,j+2w}$, $S^t_{i,j+3w}$ (resp. $C^t_{j+w}$, $C^t_{j+2w}$, $C^t_{j+3w}$), where as usual $w$ denotes the word size. For example, if $w=64$ (for \MORUS[1280]), the term $S^3_{2,0}$ is symmetrized into:
\[
S^3_{2,0} \oplus S^3_{2,64} \oplus S^3_{2,128} \oplus S^3_{2,192}.
\]

Thus, translating the trail from one of the \MiniMORUS ciphers to the corresponding full \MORUS cipher amounts to making every linear combination symmetric---indeed, that was the point of introducing \MiniMORUS in the first place.
Concretely, in order to build the full trail equation for \MORUS, we write symmetric versions of equations $A^t_i$, $B^t_i$, $C^t_i$, $D^t_i$, $E^t_i$ from \Cref{sec:minitraileq}, and then combine them in exactly the same manner as before.
This way, the biased linear combination on \MiniMORUS[1280] given in \Cref{sec:minitraileq}, namely:
\begin{align*}
&C^0_{51} \oplus C^1_{0} \oplus C^1_{25} \oplus C^1_{33} \oplus C^1_{55} \oplus C^2_{4} \oplus C^2_{7} \oplus C^2_{29} \oplus C^2_{37}\\
\oplus\; &C^2_{38} \oplus C^2_{46} \oplus C^2_{51} \oplus C^3_{11} \oplus C^3_{20} \oplus C^3_{42} \oplus C^3_{50} \oplus C^4_{24}
\end{align*}
ultimately yields the following biased symmetrized linear combination on the full \MORUS[1280]:
\begin{align*}
&C^0_{51} \oplus C^0_{115} \oplus C^0_{179} \oplus C^0_{243} \oplus C^1_{0} \oplus C^1_{25} \oplus C^1_{33} \oplus C^1_{55} \oplus C^1_{64} \oplus C^1_{89}\\
\oplus\; & C^1_{97} \oplus C^1_{119} \oplus C^1_{128} \oplus C^1_{153} \oplus C^1_{161} \oplus C^1_{183} \oplus C^1_{192} \oplus C^1_{217} \oplus C^1_{225} \oplus C^1_{247}\\
\oplus\; & C^2_{4} \oplus C^2_{7} \oplus C^2_{29} \oplus C^2_{37} \oplus C^2_{38} \oplus C^2_{46} \oplus C^2_{51} \oplus C^2_{68} \oplus C^2_{71} \oplus C^2_{93}\\
\oplus\; & C^2_{101} \oplus C^2_{102} \oplus C^2_{110} \oplus C^2_{115} \oplus C^2_{132} \oplus C^2_{135} \oplus C^2_{157} \oplus C^2_{165} \oplus C^2_{166} \oplus C^2_{174}\\
\oplus\; & C^2_{179} \oplus C^2_{196} \oplus C^2_{199} \oplus C^2_{221} \oplus C^2_{229} \oplus C^2_{230} \oplus C^2_{238} \oplus C^2_{243} \oplus C^3_{11} \oplus C^3_{20}\\
\oplus\; & C^3_{42} \oplus C^3_{50} \oplus C^3_{75} \oplus C^3_{84} \oplus C^3_{106} \oplus C^3_{114} \oplus C^3_{139} \oplus C^3_{148} \oplus C^3_{170} \oplus C^3_{178}\\
\oplus\; & C^3_{203} \oplus C^3_{212} \oplus C^3_{234} \oplus C^3_{242} \oplus C^4_{24} \oplus C^4_{88} \oplus C^4_{152} \oplus C^4_{216}
\end{align*}
We refer the reader to \Cref{sec:traileq} for the corresponding linear combination on \MORUS[640].

\subsection{Correlation of the Full Trail}

The symmetrized trail on full \MORUS may be intuitively understood as consisting of four copies of the original trail on \MiniMORUS. Indeed, the only difference between full \MORUS (for either version of \MORUS) and four independent copies of \MiniMORUS comes from register-wise rotations, which permute words within a register. But as observed in \Cref{sec/introminimorus}, register-wise rotations leave symmetric linear combinations invariant; and so, insofar as we only ever use symmetric linear combinations on all registers along the trail, register-wise rotations have no effect.

Following the previous intuition, one may expect that the weight of the symmetrized trail should simply be four times the weight of the corresponding \MiniMORUS trail, namely 64 for both \MORUS[1280] and \MORUS[640]. However, reality is a little more complex, as the symmetrized trail does not exactly behave as four copies of the original trail when one considers nonlinear terms.

To understand why that might be the case, assume a nonlinear term $S^0_{2,0} \cdot S^0_{3,0}$ arising from some part of the trail, and another term $S^0_{2,0} \cdot S^0_{3,w}$ arising from a different part of the trail (where $w$ denotes the word size). Then when we XOR the various trail fragments together, in \MiniMORUS these two terms are actually equal and will cancel out, since register-wise rotations by multiples of $w$ bits are ignored. However in the real \MORUS these terms are of course distinct and do not cancel each other.

In the actual trail for (either version of) full \MORUS, this exact situation occurs when combining trail fragments $\beta^t_i$ and $\gamma^t_i$. Indeed, $\beta^t_i$ requires approximating the term $S^t_{2,i} \cdot S^t_{3,i}$, while $\gamma^t_i$ requires approximating the term $S^t_{2,i} \cdot S^t_{3,i-w}$ (cf. \Cref{fig:trailcollision}). While in \MiniMORUS, these terms cancel out, in the full \MORUS, when adding all symmetric copies of the trail, we end up with the sum:
\begin{align}
&S^t_{2,i} \cdot S^t_{3,i} \oplus S^t_{3,i} \cdot S^t_{2,i+w}
\oplus S^t_{2,i+w} \cdot S^t_{3,i+w} \oplus S^t_{3,i+w} \cdot S^t_{2,i+2w}\notag\\
\oplus\; &S^t_{2,i+2w} \cdot S^t_{3,i+2w} \oplus S^t_{3,i+2w} \cdot S^t_{2,i+3w}
\oplus S^t_{2,i+3w} \cdot S^t_{3,i+3w} \oplus S^t_{3,i+3w} \cdot S^t_{2,i}.\label{eq:8circle}
\end{align}
It may be observed that the products occurring in the equation above involve eight terms forming a ring. The weight of this expression can be computed by brute force, and is equal to $3$.

\begin{figure}[t!]
  \substatesfalse
  % \substatesfalse to label state words and/or masks
  \centering
  \begin{subfigure}{.4\textwidth}
  \centering
  \begin{tikzpicture}[xscale=0.75,yscale=1.5]%{{{
    \printstate
    \draw[trail, beta]
      (C) -- node[right] {$i$} (lll-1)
      (lll-1) -- (tlll-1) (lll-1) -- (xor-1) (xor-1) -- (xnd-1)
      (tlll-1) -- (W-20) node[above] {$i$}
      (xor-1) -- (txor-1) %(txor-1) -- (W-21) node[above] {$i$}
      %(xnd-1) -- (and-1)
      (W40) node[below] {\phantom{$i$}}
      ;
    \draw[trail, gamma]
      %(W-21) node[above] {$i$}
      (txor-1) -- (xnd1)
      (xnd1) -- (xor1) (xor1) -- (lll1)
      %(xnd1) -- (and1)
      (xor1) -- (txor1)
      (txor1) -- (W-24) node[above] {$i$}
      (lll1) -- (W41) node[below] {$i+b_1$}
      ;
    %\draw[trail, beta,  dotted] (and-1) -- (and-1-|and1);
    %\draw[trail, gamma, dotted] (and1)  -- (and-1-|and1) node[above, black] {=};
    \node (eq) at ($(and-1|-and1)+(-1,-2.4pt)$) {=};
    \draw[beta,{Circle[sep=-1.8pt]}-]  ($(and-1)+(0,-2.4pt)$) -| (eq);
    \draw[gamma,{Circle[sep=-1.8pt]}-] ($(and1)+(0,-2.4pt)$) -- (eq);
  \end{tikzpicture}%}}}
  \caption*{\MiniMORUS: weight 0 (not 2)} %$S^{j,1}_i, S^{j,4}_i, S^{j+1,1}_{i-1}$ ($w\!=\!1$)}
  \end{subfigure}
  \qquad
  \begin{subfigure}{.4\textwidth}
  \centering
  \begin{tikzpicture}[xscale=0.75,yscale=1.5]%{{{
    \printstate
    \draw[trail, beta]
      (C) -- node[right] {$i$} (lll-1)
      (lll-1) -- (tlll-1) (lll-1) -- (xor-1) (xor-1) -- (xnd-1)
      (tlll-1) -- (W-20) node[above] {$i$}
      (xor-1) -- (txor-1) %(txor-1) -- (W-21) node[above] {$i$}
      (xnd-1) -- (and-1)
      (W40) node[below] {\phantom{$i$}}
      ;
    \draw[trail, gamma]
      %(W-21) node[above] {$i$}
      (txor-1) -- (xnd1)
      (xnd1) -- (xor1) (xor1) -- (lll1)
      (xnd1) -- (and1)
      (xor1) -- (txor1)
      (txor1) -- (W-24) node[above] {$i$}
      (lll1) -- (W41) node[below] {$i+b_1$}
      ;
    \draw[trail, beta,  dashed] (and-1.east|-tanB-1) -- (tanB-1);
    \draw[trail, gamma, dashed] (tanB-1)  -- (tanB1) -- (tanB1-|and1.east);
  \end{tikzpicture}%}}}
  \caption*{\MORUS: weight $4 \times 1$ (not $4 \times 2$)} %$S^{j,1}_i, S^{j,4}_i, S^{j+1,1}_{i-1}$ ($w\!=\!1$)}
  \end{subfigure}
  \caption{Weight of $\beta^t_i \oplus \gamma^t_i$ for \MiniMORUS and \MORUS.}
  \label{fig:trailcollision}
\end{figure}

For \MORUS[1280], since the trail fragment $\gamma^t_i$ is used four times, this phenomenon adds a contribution of $4 \cdot 3 = 12$ to the overall weight of the full trail. This results in a total weight of $4 \cdot 16 + 12 = 76$ (recall that the weight of the trail on \MiniMORUS[1280] is 16). We have confirmed this by explicitly computing the full trail equation in \Cref{sec:traileq}, and evaluating its exact weight like we did for \MiniMORUS in \Cref{sec:minibias}. That is, since the equation is quadratic, we may view it as a graph, which we split into connected components; we then compute the weight of each connected component separately by brute force, and then add up the weights of all components per the Piling-Up Lemma. Overall, the full trail equation given in \Cref{sec:traileq} yields a weight of 76 for the full trail on \MORUS[1280].

In the case of \MORUS[640], collisions between rotation constants further complicate the analysis. Specifically, when using trail fragment $\beta^t_i$, the term $S^t_{2,i} \cdot S^t_{3,i}$ occurs. As explained previously, a partial collision with the term $S^t_{2,i} \cdot S^t_{3,i-w}$ from trail fragment $\gamma^t_i$ results in \Cref{eq:8circle}. However trail fragment $\alpha^t_{i+d}$ is once used in the course of the full trail with an offset of $d = b_1+b_4-b_0-b_2$ (relative to $\gamma^t_i$), which in the case \MORUS[640] is equal to $31+13-5-7 = 0 \;\text{mod}\; 32$. This creates another term $S^t_{2,i} \cdot S^t_{3,i}$, which ultimately destroys one of the four occurrences of \Cref{eq:8circle}. Therefore, when computing the full trail equation on \MORUS[640], we get that the weight of the trail is 73 (cf. \Cref{sec:traileq}).

%TODO: check $\alpha_i \oplus \beta_i \oplus \gamma_i$ in step 2 of $\cipher{MORUS-640}$
%(maybe not relevant due to overall complexity\dots)

%%% Local Variables:
%%% TeX-master: "morusAC"
%%% End:
